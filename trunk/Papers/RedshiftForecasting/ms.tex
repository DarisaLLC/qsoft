\documentclass[12pt,preprint]{aastex}
%\documentclass[apj]{emulateapj}

%\usepackage{epsfig}
\usepackage{lscape,graphicx,rotating}
%\usepackage{apjfonts}
\usepackage{epstopdf}


%%%%%%%%%%%%%%%%%%%%%%%%%%%%%%%%%%%%%%%%%%%%%%%%%%%%%%%%%%%%%
% New Commands
%%%%%%%%%%%%%%%%%%%%%%%%%%%%%%%%%%%%%%%%%%%%%%%%%%%%%%%%%%%%%
\newcommand{\Swift}{\textit{Swift}}
\newcommand{\KW}{\textit{Konus-Wind}}
\newcommand{\HST}{\textit{Hubble Space Telescope}}
\newcommand{\hst}{\textit{HST}}
\newcommand{\gmos}{\textit{GMOS}}
\newcommand{\Bc}{\textit{B$_{\mathrm{C}}$}}
\newcommand{\Rc}{\textit{R$_{\mathrm{C}}$}}
\newcommand{\Ic}{\textit{I$_{\mathrm{C}}$}}
\newcommand{\Vc}{\textit{V$_{\mathrm{C}}$}}
\newcommand{\zG}{\textit{z}}
\newcommand{\iG}{\textit{i}}
\newcommand{\gG}{\textit{g}}
\newcommand{\ip}{\textit{i$^{\prime}$}}
\newcommand{\zp}{\textit{z$^{\prime}$}}
\newcommand{\gp}{\textit{g$^{\prime}$}}
\newcommand{\rp}{\textit{r$^{\prime}$}}
\newcommand{\J}{\textit{J}}
\newcommand{\Hm}{\textit{H}}
\newcommand{\Ks}{\textit{K$_{s}$}}
%%%%%%%%%%%%%%%%%%%%%%%%%%%%%%%%%%%%%%%%%%%%%%%%%%%%%%%%%%%%%

\begin{document}

%======================================================
% Title and Author Info
%======================================================
\title{Gamma-Ray Burst Forecasting - Redshift Inference from Early-time Metrics}
%======================================================

%======================================================
% Author List
%======================================================
\author{A.~N.~Morgan\altaffilmark{1}, S.~O.~Else\altaffilmark{1},
  and E.~A.~Einstein\altaffilmark{3}}

\altaffiltext{1}{Department of Astronomy, University of
  California, Berkeley, CA 94720-3411, USA.}
\altaffiltext{2}{Beyond the Grave}

%======================================================
% Other Misc Info
%======================================================
\slugcomment{Submitted to \textit{???}}

\shorttitle{Machine learning rocks.}
\shortauthors{Morgan \textit{et al.}}

%======================================================
% Abstract
%======================================================
\begin{abstract}
% When a Gamma-ray burst goes off, one usually needs to decide very quickly about what appropriate actions need to be taken using what little information is available at the time.  If the event is potentially interesting, say - high redshift - we'd want to make sure to use available telescope resources to observe it, while at the same time not wasting resources on events that are unlikely to be of value. In principle, indications of high redshift are present in quickly available metrics such as GRB duration, peak brightness, detections of the afterglow in the optical, and others. Other groups have used hard cuts of a small number of such parameters to make  predictions, but we aim to do better by utilizing machine learning techniques to make increasingly stronger probabilistic statements about the nature of the events (as more and more information comes in) - automatically, in real time, and using all available data In summary, given rapidly available metrics, how well can we predict the redshift for a newly discovered GRB?

As the number of observed GRBs continues to accumulate, a natural consequence 
is that the available follow-up resources need to be used more conservatively 
in order to maximize the science output from available telescope time. As 
such, it is becoming increasingly important to be able to rapidly identify 
bursts of interest using early-time metrics.  Here we present preliminary 
results making use of machine-learned classification with CART (Classification
and Regression Trees) for identification of high-redshift candidates using a 
selection of early-time and post-processed metrics from the three telescopes 
onboard \Swift.
\end{abstract}

%======================================================
% Keywords
%======================================================
\keywords{gamma-rays: bursts}

%======================================================
% Introduction
%======================================================
\section{Introduction}
\label{sec:intro}

While much work has gone into potential luminosity predictors of GRBs, a 
related but distinct problem is predicting the redshift of a new event based 
on early-time metrics.  Since the launch of the \Swift satellite, more than 170 
gamma-ray bursts have had measured redshifts, but only a handful fall into the 
highest range that allow for the probing of the earliest ages of the universe 
(Fig. 1).  With a limited budget of telescope time accessible for deep 
follow-up, it is becoming increasingly important to rapidly identify these 
bursts of interest to utilize available resources before the afterglows fade 
beyond detectability.

In principle, indications of high redshift are present in quickly available 
metrics from the three telescopes onboard \Swift (BAT, XRT, UVOT) such as GRB 
duration, peak brightness, detections of the afterglow in the optical, and 
others. Past studies have used hard cuts on a small number of these attributes 
with some success (e.g. [1], [2], [3]). In this work, we aim to improve upon 
these techniques we aim to do better by utilizing machine learning to make 
increasingly stronger probabilistic statements about the nature of the events 
(as more and more information arrives) automatically, in real time, and using 
all available data




Supervised learning, specifically classification, on the data set.

Attribute selection is in of itself a non-trivial research problem in 
statistics, and we do not attempt a full treatment here.  Instead, we have 
chosen to test attributes which we believe to be physically motivated to 
correlate with redshift (e.g. ...) or ones that have been shown in past 
studies to reveal potential correlations (e.g. ... )

Statement about independence of attributes?

Statement of correlation (Amati etc.) .. hope for a standard candle.

It should be noted that there has been significant debate in the literature as
to whether these relations are due to intrinsic physical properites of the GRBs,
or due to thresholding effects specific to the detectors (e.g. Butler 200?).  
Regardless, the nature of these relationships would not affect their utility 
as a {\it detector specific} parameter prediction tool.

%======================================================
% Data Collection
%======================================================
\section{Data Collection}
\label{sec:obs}
Data are collected in real-time directly from GCN notices and from automated 
pipelines ([1],[2]) that process and refine the data into more useful metrics.  
Tens of attributes are parsed from the various sources and collated into a 
common format.  Values with uncertainties larger than 32\% of the mean are 
removed from the sample and marked as missing values.  The attributes are 
broken up into four time periods based on their arrival time after the burst 
to simulate the clearer picture about the nature of each GRB that emerges with 
time, and our algorithm is applied at each stage.

%======================================================
% Broadband Modeling
%======================================================
\section{Machine Learning Algorithms}
\label{sec:models}
Classification and Regression Trees (CART) are a simple, flexible, and 
interpretable machine learning algorithm [6]. For the problem of 
classification, the trees generated by this algorithm form a partition of 
the predictor space. Classes are assigned to predictor vectors according to 
the partition element into which each vector falls. The partition itself is 
learned from the training data recursively. First, the root node of the 
titular tree is given a correspondence to the entire predictor space. A 
hyperplane, typically axis-aligned, is chosen to partition the space into 
two regions, each corresponding to one of the two child nodes of the tree's 
root node. Recursively, each subspace represented by a node of the tree is 
further split into two subspaces---and the node correspondingly grows two 
child nodes. The predictor type and value for each split is determined so 
as to heuristically maximize the information in the tree, and recursive 
splitting halts when, heuristically, no further information can be gained.

Our method utilizes CART to achieve the objective function of maximizing the 
number of redshift events correctly classified as high (rather than low) while
classifying at most a fixed percentage of the entire data set as 
high-redshift. This last constraint represents a restricted ability to follow 
up on high-redshift events since we consider a high-redshift classification 
equivalent to a decision to follow up.  We weight misclassification of a 
high-redshift burst as low more strongly than vice versa with under the 
assumption that the potential loss of telescope time on a relatively 
uninteresting event is less scientifically damaging than missing the highest 
redshift bursts. To achieve our objective, we evaluate CART models derived 
under a range of splitting criteria and choose that model which performs best 
according to the objective. The different node-splitting criteria are achieved
by varying the prior probability mass on each class (high or low redshift).  
To achieve an unbiased estimate of our method's performance, we report a final
evaluation obtained via ten-fold cross-validation [7].


%======================================================
\subsection{Results}
\label{sec:models:some}

While CART is particularly well-suited to deal with the normally troublesome 
problem of missing values (which are highly present in this dataset), 
additional complications persist.  The primary issue is the scarcity of high-z 
events from which to train, and further challenges lie in properly dealing 
with uncertainties in individual values, spurious values, and attribute 
selection.

As suggested by previous works, our studies indicate that the lack of UVOT 
afterglow detection (at early times, and away from the galactic plane) is by 
far the strongest predictor of redshift class.  Further, however, our 
preliminary results show that little, if any, additional information can be 
learned from the other attributes.  Our full results  on binary classification 
of redshift with CART will be presented in an upcoming paper, and future 
studies investigating the related (but more complicated) problem of CART 
regression on this dataset are to follow.



\S~\ref{sec:conclusion}.


%======================================================
%\input{tab6}
\input{./Tables/tab7.tab}
%======================================================

%=======================================================
% Discussion
%=======================================================
\section{Discussion}
\label{sec:discussion}

Discussion

%=======================================================
% Figure: EKE vs Egamma
%=======================================================
\begin{figure}[t!]
  \centerline{\plotone{./Figures/RedshiftDist.png}}
\  \caption[Redshift distribution of long-duration \Swift GRBs.]
  {Redshift distribution of long-duration \Swift GRBs.  For the purposes of 
this study,  “high” redshift is defined as those bursts with redshifts larger 
than 1-sigma above the mean, or about z~4.  In our current sample, 17 bursts 
fall into this category (green, and in inset).
 }
\label{fig:energy}
\end{figure}
%=======================================================



%=======================================================
% Conclusion
%=======================================================
\section{Conclusions}
\label{sec:conclusions}
In this work we have presented model fits to the broadband afterglows of the 

\Swift\ GRBs\,050820A, 060418, and 080319B.  

 Here is a citation: (e.g. \citealt{zwh04}). 

%======================================================
% Acknowledgments
%======================================================
\acknowledgments{
Sponsored by an NSF-CDI grant (award #0941742)  ``Real-time Classification of Massive Time-series Data Streams'' (PI: Bloom). This publication has made use of data obtained from the \Swift\ interface
of the High-Energy Astrophysics Archive (HEASARC), provided by
NASA's Goddard Space Flight Center.  }

%======================================================
% Facility Keywords
%======================================================
{\it Facilities:}
\facility{Swift (BAT, XRT, UVOT)}

%======================================================
% Bibliography
%======================================================
\begin{thebibliography}{134}
\expandafter\ifx\csname natexlab\endcsname\relax\def\natexlab#1{#1}\fi

\bibitem[{Zhang {et~al.}(2004)Zhang, Woosley, \& Heger}]{zwh04}
Zhang, W., Woosley, S.~E., \& Heger, A. 2004, \apj, 608, 365

\end{thebibliography}


%======================================================
% The End
%======================================================
\end{document}

